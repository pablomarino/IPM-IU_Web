\documentclass[11pt,a4paper]{article}

\usepackage[spanish]{babel}
\usepackage[utf8]{inputenc}
\usepackage{url}


\title{Práctica 3 - Interfaces de Usuaria en aplicaciones Web}
\author{Interfaces Persona Máquina}
\date{Curso 16/17}

\renewcommand{\abstractname}{Objetivos}


\begin{document}


\maketitle

\begin{abstract}
  Aplicar los conocimientos adquiridos sobre el desarrollo de
  interfaces de usuaria para aplicaciones Web.
\end{abstract}


%%%%%%%%%%%%%%%%%%%%%%%%%%%%%%%%%%%%%%%%%%%%%%%%%%%%%%%%%%%%%%%%%%%%%%%%%%%
\section{Descripción}

Tendrás que rediseñar una interface de una aplicación web existente
para que la experiencia con dispositivos móviles sea más
satisfactoria, sin que ello suponga un perjuicio sobre las capacidades
actuales.

Estamos trabajando con una aplicación existente, pero no puedes
acceder al código de la misma, ni integrar tu parte de ninguna
manera. Por tanto, únicamente tienes que implementar la parte de
interface de usuario, y el backend de la aplicación será fingido
(\emph{mock})\footnote{Puedes usar datos ficticios o copiar los que
  presenta la aplicación real.}.

Los siguientes apartados describen los \emph{sprints} que debes
realizar según la planificación establecida.


\subsection{Requisitos no funcionales}
\begin{itemize}
\item La implementación se realizará siguiendo las normas del W3C.
\end{itemize}


%%%%%%%%%%%%%%%%%%%%%%%%%%%%%%%%%%%%%%%%%%%%%%%%%%%%%%%%%%%%%%%%%%%%%%%%%%%
\section{Sprint 1}

En este \emph{sprint} debes realizar los siguientes pasos:

\begin{enumerate}
\item Examina la aplicación:
  \url{http://competiciones.fgpatinaxe.gal/}, ignorando la intranet.
  Documenta el diseño de la interface gráfica actual.

  Puedes emplear el formato de tu elección para documentar el diseño.

\item Documenta los posibles puntos débiles y/o mejoras.

\item Valida todos los pasos anteriores. A continuación asígnale al
  último commit del repositorio la etiqueta \texttt{sprint1}.

\item Valida el contenido del repositorio remoto\footnote{HINT:
    después de clonarlo, puedes hacer un reset a la etiqueta
    \texttt{sprint1} (\texttt{git reset --hard sprint1})}.
\end{enumerate}


%%%%%%%%%%%%%%%%%%%%%%%%%%%%%%%%%%%%%%%%%%%%%%%%%%%%%%%%%%%%%%%%%%%%%%%%%%%
\section{Sprint 2}

En este \emph{sprint} debes realizar los siguientes pasos:

\begin{enumerate}
\item Siguiendo las ideas de las metodologías \emph{mobile first},
  diseña una nueva interface que se adapte de forma conveniente para
  su uso con dispositivos móviles y desktops.

\item Implementa el diseño. Dentro de las normas del W3C debes escoger
  \texttt{html5} y \texttt{css3}.

\item Valida todos los pasos anteriores, en especial el funcionamiento
  de tu implementación y el cumplimiento de los estándares
  señalados. A continuación asígnale al último commit del repositorio
  la etiqueta \texttt{sprint2}.

\item Valida el contenido del repositorio remoto.
\end{enumerate}


%%%%%%%%%%%%%%%%%%%%%%%%%%%%%%%%%%%%%%%%%%%%%%%%%%%%%%%%%%%%%%%%%%%%%%%%%%%
\section{Sprint 3}

En este \emph{sprint} debes realizar los siguientes pasos:

\begin{enumerate}
\item Documenta las posibles fallos de tu implementación con respecto
  a las normas WAI-ARIA.

\item Soluciona los problemas detectados.

\item Valida todos los pasos anteriores. A continuación asígnale al último commit del
  repositorio la etiqueta \texttt{sprint3}.

\item Valida el contenido del repositorio remoto.
\end{enumerate}






\end{document}
